\documentclass[../main.tex]{subfiles}
\section{Handed-out Materials}
The handed-out framework comes with a convenient make file.
The make file had targets for building the file for uploading, the ELF file used with GDB and for actually uploading the code to the micro-controller.

A \emph{efm32gg.h} file that contained all the registries that will be needed in this exercises were given since the coding would be done in C.
The skeleton code is spread across 4 \emph{.c} files.
The files were \emph{dac.c} \emph{ex2.c} \emph{gpio.c} \emph{interrupt\_handler.c} and \emph{timer.c}.
The \emph{dac.c} file has the sole purpose of enabling the digital/analogue converter.

\emph{timer.c} is a file that enables the timer interrupts.
\emph{gpio.c} is a file that initializes the GPIO registers.
\emph{interrupt\_handler.c} is where the GPIO interrupts and the timer interrupts are found.
Finally the \emph{ex2.c} is where the main function is and where the setup functions are ran.
