\section{Introduction}
Energy efficient programming is a very important field within computer science due to the expanding market of battery dependent devices.
Energy efficient programming means that we try to use as little power as possible.
This would in turn increase the battery life of portable devices.
Also if a micro controller is used in large quantities lets say a factory, then each percent power saved could have a large impact on the power bill.

Excercise 1 in TDT4258 required the team to write a program in assembly that would allow the user to use the gamepad buttons to control the LEDs on the gamepad.
Among the learning outcomes where \emph{interrupt handling} and \emph{simple energy optimiztions}.
With these lenient requirements, the team decided that a simple program would suffice, and focus should be put on gaining the learning outcome of the excercise.
This report outlines what the team learned, and how the program was developed.
